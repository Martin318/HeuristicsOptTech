\documentclass [11pt]{article}
\usepackage[left=1in, right=1in, top=1in, bottom=1in]{geometry}
\usepackage{titling}
\usepackage{lipsum}
\usepackage[utf8]{inputenc}
\usepackage{geometry}
\usepackage{tabu}
\usepackage{float}
\usepackage{caption}
\usepackage{rotating}
\usepackage{titlesec}
\usepackage{titling}
\usepackage{graphicx}
\usepackage{amsmath}
\usepackage{amssymb}
\usepackage{DejaVuSansMono}
\usepackage[parfill]{parskip}
\usepackage[autostyle=false, style=english]{csquotes}
\MakeOuterQuote{"}
\geometry{total={240mm,320mm},
          left=25mm,right=25mm,%
          bindingoffset=0mm, top=25mm,bottom=25mm}


\floatstyle{boxed} 
\restylefloat{figure} % Makes floats have a frame

% \renewcommand{\ttdefault}{DejaVuSansMono-TLF}
% \renewcommand{\rmdefault}{lmss}
\linespread{1}
\newcommand{\linia}{\rule{\linewidth}{0.2pt}}

\titlespacing\section{0pt}{12pt plus 4pt minus 2pt}{0pt plus 2pt minus 2pt}

\makeatletter
\def\@maketitle{%
\nopagebreak[4]
  \vspace*{-\topskip}      % remove the initial space
  \begingroup\centering    % instead of \begin{center}
  \let \footnote \thanks
  \hrule height \z@        % to avoid the insertion of lineskip glue
    {\huge \@title \par}%
    \vskip 0.5em 
    {\large
      \lineskip .5em 
      \begin{tabular}[t]{c}%
        \@author 
      \end{tabular}\par}%
    % \vskip 1em 
    % {\large \@date}%
  \par\endgroup            % instead of \end{center}
  \let\endtitlepage\relax\let\endtitlepage\relax
  \vskip 1.5em             % <--- modify this to adjust the separation
\large}
\makeatother


\begin{document}
\inputencoding{utf8}
\title{ Heuristic Opt. Techniques - Assignment 4 Report}
\author{ Martin Blöschl and Cem Okulmus }

\maketitle
\thispagestyle{empty}


\section{Implementation}
We chose to implement a Genetic Algorithm (GA) for this assignment. For this we set out to find an encoding that is easier for the GA to work with, and then implement the different operators for Selection, Recombination, Mutation and Replacement. 

\subsection{Encoding}

Since the spine-order is already stored as just an array of integers, we simply thought of it as part of the ``chromosome''. For the edge partitions, we encode a list of pairs of edges (which itself is an integer pair) and a page (as an integer). For decoding it, we create the edge partition as an array of the length of the pages, each cell containing the corresponding list of edges. 

\subsection{Fitness function}

We wanted our fitness function to depend only on the crossings of the individual. For genetic algorithms, it is usually the case that higher fitness is better. Thus, using just the number of crossings would not have the desired effect (since more crossings are bad). We first compute for a given number of edges, the maximal number of possible crossings, regardless of page size. This can be thought of as if every edge would cross all others. So 
\[
	maxCrossings(sol) = \frac{|sol.edges| * (|sol.edges| - 1)}{2} 
\]
Where $|X|$ denotes cardinality of a set $X$ and $sol.edges$ is the list of edges in a solution.

The fitness function f is then 
\[
	f(sol) = maxCrossings(sol) - crossings(sol)
 \]

 where $crossings(sol)$ computes the actual crossings in the given solution $sol$.

\subsection{Initial population}

First, we used some solutions constructed by our advanced construction heuristics (see previous assignment reports) as initial population. However it turned out that those solutions were mostly very similar, and thus the initial population was not diverse enough. We then used a set of purely random initial solutions (random page assignment, random spine order). This turned out to be worse in the beginning, but eventually after some generations lead to better results.


\subsection{Selection}

We implemented a simple selection method that just selects the better half of the population for recombination. "Better" here means higher fitness (=lower unfitness, see section fitness function). To get more precise, we sort the population by crossings ascending and pick all individuals from 0 to half the size of the population (rounded down).

\subsection{Recombination}
The recombination creates for two ``parent'' solutions a ``child'', by first creating a new spine order. For this end, a mapping between the existing two spine orders is considered. This mapping can be thought of as a permutation. We take the first cycle in this permutation (which could span the whole permutation) and apply it to either one, thus creating a new spine order for the ``child''. 

For the edge partition, we take for half the edges the page assignments from ``parent'' one and for the other half of the edges the assignments from ``parent'' two. No particular order is considered for the edges.

\subsection{Mutation}

We mutate approximately 33\% of the individuals and from those exactly one gene. Of course we could spend some time optimizing these parameters. In the future, we could try mutating less individuals but in those selected for mutation we mutate multiple genes, or we could try mutating even more individuals. Since our parameter setting (33\%, mutate one gene) turned out to give quite good results, we did not spend additional time adjusting these parameters.

The mutation of a gene is in our case changing the page assignment of an edge. Every edge must be assigned to one page. If such an assignment is mutated, we assign this edge to a random other page.

\subsection{Replacement}
We decided early on to stick with a form of generational replacement, but where only the better  half  (w.r.t. fitness) of the previous generation is replaced. The size of the populations always stays the same. Since we always replace the worse half of the population, this implies that the fitness of the generation (fitness of the best individual) monotonically increases. The best individual is never dropped (except if all individuals have the same fitness of course).

\section{Evaluation}

The evaluation was run, once again, on the eowyn cluster. For instances 1 through 5, we used a population size of 10000 and 200 generations. For instances 6 through 10, we reduced the population size to 1000 and generations to 100. This was for performance reasons, as the increase in memory size reduced the performance drastically. The results for 10 runs each can be seen on table 1.


\section{Observation}
For the smaller instances (1 through 5), the GA almost always found either optimal or compared with the other solution submissions from the other groups, the best known solutions (the exception being 5). For the larger instances this was not the case. The smaller population size reduced the variation and therefore effectiveness of the meta heuristic. 

\section{Conclusion}

Despite our very simple implementation, the Generic Algorithm works surprisingly well and in some instances (in particular the small instances) even better than the local search procedures from the previous assignments. This could be explained by the fact that we generate random initial solutions and with a large population and a lot of generations we can cover a large part of the search space. In larger instances, this is computationally infeasible and thus the local search methods dominate here. 

\newpage


\begin{sidewaystable}
  \centering
 \label{fig:results}
  \everyrow{\hline}
  \begin{tabu} {| l |r | r |r |r|r |}
  Instance        & Global best   & Mean (10 runs)   & std. dev. & runtime mean (sec.) & runtime std. dev.    \\ 
  automatic-1     & 9             & 9                   & 0.0           &  11.37 & 0.36       \\ 
  automatic-2     & 0             & 0.2                & 0.42          &  14.19 & 0.89      \\ 
  automatic-3     & 26            & 41.7                & 6.98          & 37.66 &1.38       \\ 
  automatic-4     & 0             & 0.3                 & 0.67          &  27.14 & 0.68       \\ 
  automatic-5     & 2             & 5.6                & 2.22          &  34.76 & 0.68      \\ 
  automatic-6*     & 8360244       & 8404827           & 39653.48      &  900 & 0        \\ 
  automatic-7*     & 92502        & 100305            & 7597.2       &  146.93 & 7.04        \\ 
  automatic-8*     & 947961        & 989561              & 30022.91       &  900 &0        \\ 
  automatic-9*     & 1347025        & 1453652              & 82271.8        &  900 &0        \\ 
  automatic-10*    & 202479         & 207943               & 4081.1        &  900 &0        \\ 
\end{tabu}
\caption{Test results for GA. Entries marked with Star (*) use different population sizes as described in the report. Also there was a hard time out of 15 minutes (900 seconds) at which point the last result was output} 
\end{sidewaystable}


\end{document}