\documentclass [11pt]{article}
\usepackage[left=1in, right=1in, top=1in, bottom=1in]{geometry}
\usepackage{titling}
\usepackage{lipsum}
\usepackage[utf8]{inputenc}
\usepackage{geometry}
\usepackage{tabu}
\usepackage{float}
\usepackage{caption}
\usepackage{rotating}
\usepackage{titlesec}
\usepackage{titling}
\usepackage{graphicx}
\usepackage{amsmath}
\usepackage{amssymb}
\usepackage{DejaVuSansMono}
\usepackage[parfill]{parskip}
\usepackage[autostyle=false, style=english]{csquotes}
\MakeOuterQuote{"}
\geometry{total={240mm,320mm},
          left=25mm,right=25mm,%
          bindingoffset=0mm, top=25mm,bottom=25mm}


\floatstyle{boxed} 
\restylefloat{figure} % Makes floats have a frame

% \renewcommand{\ttdefault}{DejaVuSansMono-TLF}
% \renewcommand{\rmdefault}{lmss}
\linespread{1}
\newcommand{\linia}{\rule{\linewidth}{0.2pt}}

\titlespacing\section{0pt}{12pt plus 4pt minus 2pt}{0pt plus 2pt minus 2pt}

\makeatletter
\def\@maketitle{%
\nopagebreak[4]
  \vspace*{-\topskip}      % remove the initial space
  \begingroup\centering    % instead of \begin{center}
  \let \footnote \thanks
  \hrule height \z@        % to avoid the insertion of lineskip glue
    {\huge \@title \par}%
    \vskip 0.5em 
    {\large
      \lineskip .5em 
      \begin{tabular}[t]{c}%
        \@author 
      \end{tabular}\par}%
    % \vskip 1em 
    % {\large \@date}%
  \par\endgroup            % instead of \end{center}
  \let\endtitlepage\relax\let\endtitlepage\relax
  \vskip 1.5em             % <--- modify this to adjust the separation
\large}
\makeatother


\begin{document}
\inputencoding{utf8}
\title{ Heuristic Opt. Techniques - Assignment 4 Report}
\author{ Martin Blöschl and Cem Okulmus }

\maketitle
\thispagestyle{empty}


\section{Implementation}
We chose to implement a Genetic Algorithm (GA) for this assignment. For this we set out to find an encoding that is easier for the GA to work with, and then implement the different operators for Selection, Recombination, Mutation and Replacement. For the last element, we decided early on to stick with a form of generational replacement, but where only the better  half  (w.r.t. fitness) of the previous generation is replaced. 

\subsection{Encoding}

Since the spine-order is already stored as just an array of integers, we simply thought of it as its own ``chromosome''. For the edge partitions, we encode a list of pairs of edges (which itself is an integer pair) and a page (as an integer). For decoding it, we create the edge partition as an array of the length of the pages, each cell containing the corresponding list of edges.


\subsection{Initial population}

First, we used some solutions constructed by our advanced construction heuristics (see previous assignment reports) as initial population. However it turned out that those solutions were mostly very similar, and thus the initial population was not diverse enough. We then used a set of purely random initial solutions (random page assignment, random spine order). This turned out to be worse in the beginning, but eventually after some generations lead to better results.



\subsection{Selection}

We implemented a simple selection method that just selects the better half of the population for recombination. "Better" here means less crossings. To get more precise, we sort the population by crossings ascending and pick all individuals from 0 to half the size of the population (rounded down).

\subsection{Recombination}



\subsection{Mutation}

We mutate approximately 33\% of the individuals and from those exactly one gene. Of course we could spend some time optimizing these parameters. In the future, we could try mutating less individuals but in those selected for mutation we mutate multiple genes, or we could try mutating even more individuals. Since our parameter setting (33\%, mutate one gene) turned out to give quite good results, we did not spend additional time adjusting these parameters.

The mutation of a gene is in our case changing the page assignment of an edge. Every edge must be assigned to one page. If such an assignment is mutated, we assign this edge to a random other page.

\subsection{Replacement}

As stated above, we replace the worse half of our population with the children of the selected population. The size of the populations always stays the same.

\section{Evaluation}

\section{Observation}


\end{document}