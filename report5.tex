\documentclass [11pt]{article}
\usepackage[left=1in, right=1in, top=1in, bottom=1in]{geometry}
\usepackage{titling}
\usepackage{lipsum}
\usepackage[utf8]{inputenc}
\usepackage{geometry}
\usepackage{tabu}
\usepackage{float}
\usepackage{caption}
\usepackage{rotating}
\usepackage{titlesec}
\usepackage{titling}
\usepackage{graphicx}
\usepackage{amsmath}
\usepackage{amssymb}
\usepackage{DejaVuSansMono}
\usepackage[parfill]{parskip}
\usepackage[autostyle=false, style=english]{csquotes}
\MakeOuterQuote{"}
\geometry{total={240mm,320mm},
          left=25mm,right=25mm,%
          bindingoffset=0mm, top=25mm,bottom=25mm}


\floatstyle{boxed} 
\restylefloat{figure} % Makes floats have a frame

% \renewcommand{\ttdefault}{DejaVuSansMono-TLF}
% \renewcommand{\rmdefault}{lmss}
\linespread{1}
\newcommand{\linia}{\rule{\linewidth}{0.2pt}}

\titlespacing\section{0pt}{12pt plus 4pt minus 2pt}{0pt plus 2pt minus 2pt}

\makeatletter
\def\@maketitle{%
\nopagebreak[4]
  \vspace*{-\topskip}      % remove the initial space
  \begingroup\centering    % instead of \begin{center}
  \let \footnote \thanks
  \hrule height \z@        % to avoid the insertion of lineskip glue
    {\huge \@title \par}%
    \vskip 0.5em 
    {\large
      \lineskip .5em 
      \begin{tabular}[t]{c}%
        \@author 
      \end{tabular}\par}%
    % \vskip 1em 
    % {\large \@date}%
  \par\endgroup            % instead of \end{center}
  \let\endtitlepage\relax\let\endtitlepage\relax
  \vskip 1.5em             % <--- modify this to adjust the separation
\large}
\makeatother


\begin{document}
\inputencoding{utf8}
\title{ Heuristic Opt. Techniques - Assignment 5+6 Report}
\author{ Martin Blöschl and Cem Okulmus }

\maketitle
\thispagestyle{empty}


\section{Assignment 5}


\subsection{Implementation}

In the previous exercise, we developed a genetic algorithm. As stated in the previous report, our algorithm provided quite good results by itself. However, we wanted to improve the results by combining the genetic algorithm with local search.

Our genetic algorithm was implemented to be as simple and fast as possible. This allowed us to have a large population and compute a lot of generations. By incorporating local search, we want to improve the already very diverse solutions by reaching optimality in a certain neighbourhood.

Our Metaheuristic works in the follwing way: First, we execute the genetic algorithm to compute solutions. The best solution will then be used as a starting solution for local search. We can use any local search neighbourhood and step function from the previous exercises. Local search will be executed and the best solution from the genetic algorithm will be improved further. The result of the local search will  be a solution that (if first improvement or best improvement is used as step function) is as least as good as the solution from our genetic algorithm and also a local optimum in a certain neighbourhood.


\subsection{Experimental setup}


We tested our implementation locally on a PC with the following specification: Intel i7-4500U processor, 8 GB of RAM, Ubuntu 16.10 64 bit. Note that we have only tested our implementation with one configuration. The reason for this is that we tested different parameter configurations for the assignment 6 anyway. The results of this section are meant to be "reasonable default values" that we can compare to the results from the last assignments.
The configuration is the following:

XXXXXXXXXXXXX;XXXXXXXX;XXXXXXXXXXXXXXX;XXXXXXXXXXXXXXXXXXXXXX;X



\subsection{Results + Runtimes}



\begin{table}[]
\centering
\caption{The results and runtimes (rt) using default values of the algorithm.}
\label{my-label}
\begin{tabular}{l|l|l|l|l|l|l|l|l|l|l}
               &  1 &  2 &  3 &  4 &  5 &  6 &  7 &  8 &  9 &  10                    \\ \hline
Best           & 9          & 0          & 43         & 0          & 4          & 8416290    & 82565      & 1107609    & 1582522    & \multicolumn{1}{l|}{212878}   \\ \hline
Mean           & 9.2        & 0          & 50.7       & 0.6        & 7.2        & 8441085.9  & 90530.5    & 1112839.1  & 1596328.5  & \multicolumn{1}{l|}{215352.8} \\ \hline
Std. D & 0.34       & 0          & 2.84       & 0.04       & 1.46       & 34687.3    & 6546.04    & 19450.45   & 90587.87   & \multicolumn{1}{l|}{2154.52}  \\ \hline
Mean rt   & 14.99      & 21.74      & 78.21      & 54.3       & 69.23      & 900        & 369.21     & 900        & 900        & \multicolumn{1}{l|}{900}      \\ \hline
Std. D rt   & 1.36       & 1.87       & 2.03       & 1.10       & 1.46       & 0          & 10.88      & 0          & 0          & \multicolumn{1}{l|}{0}        \\ \cline{2-11} 
\end{tabular}
\end{table}



\subsection{Comparison}


Looking at our results in Table 1 and the results from the last exercise, we can see that the number of crossings are very similar. We also see a slight increase of runtime, since our genetic algorithm was extended with a local search.

We will see if we can further improve our results by optimizing our parameters.


\section{Assignment 6}



\end{document}